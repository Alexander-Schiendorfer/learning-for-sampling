\documentclass[conference]{IEEEtran}

\ifCLASSINFOpdf
\else
\fi

\usepackage[T1]{fontenc}
\usepackage[utf8]{inputenc}
\usepackage{graphicx}
\usepackage{amsmath}
\usepackage{amssymb}
\usepackage{color}
\usepackage{bm}
\usepackage{listings}
\usepackage{tikz}
\usetikzlibrary{shapes,snakes}
\usepackage{multirow}
\usepackage{rotating}
\usepackage{soul}
\usepackage{url}
\usepackage{pgf}
\usepackage{tabularx}
\usepackage{todonotes}
\usepackage{flushend}

%% TikZ
\usepackage{tikz}
\usetikzlibrary{arrows}
\usetikzlibrary{calc}
\usetikzlibrary{trees}
\usepackage{booktabs}

% custom definitions
\definecolor{highlightblue}{rgb}{0.1,1.0,1.0}
\definecolor{highlightgreen}{rgb}{0.5,1.0,0.0}
\definecolor{highlightyellow}{rgb}{1.0,1.0,0.1}

% custom commands
\newcommand{\todot}[1]{\sethlcolor{highlightyellow} \hl{\textbf{TODO:} #1}}
\newcommand{\note}[1]{\sethlcolor{highlightblue} \hl{\textbf{NOTE:} #1}}
\newcommand{\noteg}[1]{\sethlcolor{highlightgreen} \hl{\textbf{NOTE:} #1}}
\newcommand{\eval}[1]{\sethlcolor{red} \hl{\textbf{EVAL:} #1}}
\newcommand{\impl}[1]{\sethlcolor{orange} \hl{\textbf{EVAL:} #1}}

\newcommand{\setof}[1]{\ensuremath{\{#1\}}}
\newcommand{\tupleOf}[1]{\ensuremath{\langle#1\rangle}}
\newcommand{\twopartdef}[4]
{ \left\{
\begin{array}{ll}
#1 \mbox{if } #2 \\
#3 \mbox{if } #4
\end{array}
\right.
}

% custom notation of the optimization problem etc
% general constraint model
\newcommand{\AnAgent}{a}
\newcommand{\ActualSym}{A}
\newcommand{\Actual}[2]{\ActualSym_{#1}[#2]}

\newcommand{\Proposed}[2]{\ProposalSym_{#1}[#2]}
\newcommand{\ActualPred}[2]{\widehat{\ActualSym}_{#1}[#2]}
\newcommand{\ActualTxt}[3]{\ActualSym^{#1}_{#2}[#3]}
\newcommand{\ActualMin}[2]{\ActualTxt{\mathrm{min}}{#1}{#2}}
\newcommand{\ActualMax}[2]{\ActualTxt{\mathrm{max}}{#1}{#2}}
\newcommand{\ActualMinConstant}[1]{\ActualSym^{\mathrm{min}}_{#1}}
\newcommand{\ActualMaxConstant}[1]{\ActualSym^{\mathrm{max}}_{#1}}

\newcommand{\MaxContribution}[1]{\ActualSym^{\mathrm{max}}_{#1}}
\newcommand{\MinContribution}[1]{\ActualSym^{\mathrm{min}}_{#1}}
\newcommand{\AvppSet}{\mathcal{I}}
\newcommand{\AnAvpp}{\lambda}
\newcommand{\TlAvpp}{\Lambda}
\newcommand{\Agents}{\mathcal{A}}
\newcommand{\Subsystem}{{\Agents_{\AnAvpp}}}
\newcommand{\env}{\textit{env}}
\newcommand{\MaxDeltaAgent}[1]{\Delta \ActualSym^{\max}_{#1}}

% more model commands
\newcommand{\Production}{S}
\newcommand{\DemandSym}{A_\env}
\newcommand{\ResidualLoadSym}{E_\env}
\newcommand{\ResidualLoad}[1]{\ResidualLoadSym_{#1}}
\newcommand{\AssResidualLoad}[2]{\AgentPower{#1}{#2}}
\newcommand{\IndAssResidualLoad}[2]{\AssResidualLoad{#1}^{#2}}

%\newcommand{\ScheduleSym}{S}
%\newcommand{\Schedule}[2]{\ScheduleSym_{#1}^{#2}}

\newcommand{\CostsObjective}{\Gamma}
\newcommand{\ViolationObjective}{\Delta}

\newcommand{\CostFunctionSym}{\kappa}
\newcommand{\CostFunctionSynt}[1]{\CostFunctionSym_{#1}}
\newcommand{\CostFunction}[2]{\CostFunctionSynt{#1}(#2)}
\newcommand{\ChangeSpeedSym}{\overrightarrow{\ActualSym}}

\newcommand{\ChangeSpeedMin}[1]{\ChangeSpeedSym^{\mathrm{min}}_{#1}}
\newcommand{\ChangeSpeedMax}[1]{\ChangeSpeedSym^{\mathrm{max}}_{#1}}

\newcommand{\tnow}{t_{\mathrm{now}}}
\newcommand{\tnext}{t_{\mathrm{next}}}
\newcommand{\HorizonMax}{H}
\newcommand{\Time}{\mathcal{T}}
\newcommand{\Horizon}{\mathcal{W}}
\newcommand{\Demand}[1]{\DemandSym[{#1}]}
\newcommand{\AgentPower}[2]{\Production_{#2}[#1]}
\newcommand{\StateTimeAgent}[2]{\Production_{#1}^{#2}}
\newcommand{\Power}[2]{\Production_{\mathrm{#1}}^{#2}}
\newcommand{\ListSym}{L}
\newcommand{\ListProduction}[2]{\ListSym^{#1}_{#2}}
\renewcommand{\Downarrow}{{\downarrow}}



% References
\newcommand{\sref}[1]{Sect.~\ref{#1}}
\newcommand{\fref}[1]{Fig.~\ref{#1}}
\newcommand{\aref}[1]{Alg.~\ref{#1}}
\newcommand{\tref}[1]{Table~\ref{#1}}
\newcommand{\eref}[1]{Eq.~\ref{#1}}
\newcommand{\lref}[1]{Listing~\ref{#1}}

\newcommand\blfootnote[1]{%
  \begingroup
  \renewcommand\thefootnote{}\footnote{#1}%
  \addtocounter{footnote}{-1}%
  \endgroup
}


\usepackage{subcaption}

\begin{document}
%\title{Autonomous Scheduling in a Hierarchical System of\\Self-Organizing Autonomous Virtual Power Plants}
\title{Improving Model Abstraction by Active Learning*}

\author{
\IEEEauthorblockN{Alexander Schiendorfer, Christoph Lassner, and Wolfgang Reif}
\IEEEauthorblockA{
Universität Augsburg, Germany\\
E-Mail: \{alexander.schiendorfer, christoph.lassner, reif\}@informatik.uni-augsburg.de}
}

\maketitle

\begin{abstract}
Organizational structures such as hierarchies provide an effective means to
deal with the increasing complexity found in large-scale energy systems that 
results from uncertainties in nature as well as computational efforts in scheduling. 
Abstraction-based methods provide a way to calculate a simpler behavior model 
to be used in optimization in lieu a combination of a set of behavior models.
In particular, functional dependencies over the combinatorial domain 
are approximated by repeatedly sampling input-output pairs
and substituting the actual function by piecewise linear functions. However, if
the selected input-output pairs are selected in a weakly informative way, the resulting abstracted
optimization problem introduces severe errors in quality as well as bad runtime performance.
We therefore propose to apply methods from active learning based on decision trees for regression
to search for informative
input candidates to sample and present preliminary results that motivate further research. 

\blfootnote{*This research is partly sponsored by the research unit \emph{OC-Trust} (FOR 1085) of the German Research Foundation.}
\end{abstract}

%\begin{IEEEkeywords}
%\noteg{Keywords (not necessary)}
%\end{IEEEkeywords}

\section{Hierarchical Distributed Energy Management}
Future energy systems move from systems of relatively few centrally organized units
providing most of the power demanded by consumers to many highly distributed units.
To deal with the resulting complexity in scheduling and controlling power plants in the face of 
uncertainties introduced by nature and technical deficiencies, hierarchical organizations 
that form autonomously can be employed. To achieve a reduction of complexity in the optimization 
problem to be solved by the overall system, techniques are borrowed from abstraction. 
In particular, functional dependencies over a combinatorial input domain stemming from the
aggregate of underlying agents are approximated by repeatedly sampling input-output pairs
and substituting the actual function by piecewise linear functions. However, if
the selected input-output pairs are selected in a weakly informative way, the resulting abstracted
optimization problem introduces severe errors in quality as well as bad runtime performance.
We therefore propose the use of methods from active learning to search for informative
input candidates to sample and present preliminary results that motivate further research. 

In general, the problem to be solved is a hierarchical resource allocation problem~\cite{VanZandt1995}.

\begin{eqnarray}
\label{eq:csop-scheduling-hierarchy} \textstyle
		& \underset{\AgentPower{t}{\AnAgent}}{\operatorname{minimize}} & 
		\alpha_{\ViolationObjective} \cdot \ViolationObjective + \alpha_{\CostsObjective} \cdot \CostsObjective \\
		& \operatorname{subject\ to} & \forall a \in \Subsystem, \forall t \in \Horizon : \exists [x , y] \in \ListProduction{t}\AnAgent : x \leq \AgentPower{t}{\AnAgent} \leq y\text{,} \nonumber \\
		& & \ChangeSpeedMin\AnAgent \left(\AgentPower{t-1}\AnAgent\right) \leq \AgentPower{t}{\AnAgent} \leq \ChangeSpeedMax\AnAgent\left(\AgentPower{t-1}\AnAgent\right) \nonumber\\
		& & \textstyle \text{with } \ViolationObjective = \sum_{t \in \Horizon} \left|\Production_\Subsystem[t] - \AssResidualLoad{t}{\AnAvpp} \right| \text{, } \nonumber \\
		& & \textstyle \text{and } 
		\CostsObjective = \sum_{a \in \Subsystem,\ t \in \Horizon} \CostFunction\AnAgent{\AgentPower{t}{\AnAgent}} \nonumber
\end{eqnarray}

We propose to solve it using an approach based on self-organization:
\begin{itemize}
\item A so-called ``regio-central'' approach: agents transfer models to their local supervisor who, at meso-level,
centrally optimizes the allocation~\cite{Schiendorfer2014, SchiendorferSyn2014}
\item An auction-based decentralized approach~\cite{Anders-TAAS-2015}
\end{itemize}

We align the minimal set of constraints along the physical requirements that 
power plants impose: a minimal and maximal power boundary, discontinuity by 
the ability to be switched off as well as a function limiting the possible
change in production over a certain period of time. The latter function
might depend on the type of an agent as well as the current contribution.

\begin{figure*}
        \centering
        \begin{subfigure}[b]{0.5\textwidth}
        \centering
                \includegraphics[width=\textwidth]{img/costs}
                \caption{Accuracy affected by the number of sampling points selected.}
                \label{fig:sampling}
        \end{subfigure}%
        ~ \qquad%add desired spacing between images, e. g. ~, \quad, \qquad, \hfill etc.
          %(or a blank line to force the subfigure onto a new line)
        \begin{subfigure}[b]{0.45\textwidth}
        \centering
                \includegraphics[width=\textwidth]{img/gaussProc}
                \caption{A probabilistic regression model allows to quantify uncertainty at given points in the domain of a learned function.}
                \label{fig:mouse}
        \end{subfigure}
\end{figure*}

\begin{figure*}
		\centering
		 \includegraphics[width=0.65\textwidth]{img/avpp-hierarchical-system-structure.pdf}
			\caption{Hierarchical system structure of a future autonomous power management system: Prosumers are structured into systems of systems represented by AVPPs acting as intermediaries, thereby decreasing the complexity of control and scheduling. AVPPs can be part of other AVPPs.}
			%and participate in the power market.}
		\label{fig:avpp-hierarchical-system-structure}
\end{figure*}


\section{Issues with Model Abstraction}

\section{Improving Sampling Point Selection}

\section{Evaluation}
We investigate the effects of selecting a particular set of 
sampling points for one group that could have emerged as part
of a self-organization process.


%\section*{Acknowledgment}
%This research is partly sponsored by the research unit \emph{OC-Trust} (FOR 1085) of the German Research Foundation.

\bibliographystyle{IEEEtran}
\bibliography{saos}

\end{document}
